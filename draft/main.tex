\documentclass[reprint,aps,superscriptaddress]{revtex4-2}
%\documentclass[12pt]{article}
\usepackage{aps}
\usepackage{newtxtext}




\begin{document}
%%%%%%%%%%%%%%%%%%%%%%%%%%%%%%%%%%%%%%%%%%%%%%%%%%%%%%%%%
%%%%%%%%%%%%%%%%%%%%%%%%%%%%%%%%%%%%%%%%%%%%%%%%%%%%%%%%%
%%%%%%%%%%%%%%%%%%%%%%%%%%%%%%%%%%%%%%%%%%%%%%%%%%%%%%%%%
\title{
Quantum Geometry and Nonlinear Electronic Transport in Kagome Bilayers
}
%%%%%%%%%%%%%%%%%%%%%%%%%%%%%%%%%%%%%%%%%%%%%%%%%%%%%%%%%
%%%%%%%%%%%%%%%%%%%%%%%%%%%%%%%%%%%%%%%%%%%%%%%%%%%%%%%%%
%%%%%%%%%%%%%%%%%%%%%%%%%%%%%%%%%%%%%%%%%%%%%%%%%%%%%%%%%
\author{Nabil Atlam}
\affiliation{Department of Physics,
Northeastern University,
Boston MA, USA}




\begin{abstract}

In this paper, we investigate the quantum metric-induced nonlinear transport phenomena on bilayer Kagome lattice systems in the non-interacting limit in the tight-binding approach. 

\cite{tangHighTemperatureFractionalQuantum2011a}

\end{abstract}

\maketitle
%%%%%%%%%%%%%%%%%%%%%%%%%%%%%%%%%%%%%%%%%%%%%%%%%%%%%%%%%
%%%%%%%%%%%%%%%%%%%%%%%%%%%%%%%%%%%%%%%%%%%%%%%%%%%%%%%%%
%%%%%%%%%%%%%%%%%%%%%%%%%%%%%%%%%%%%%%%%%%%%%%%%%%%%%%%%%

\section{Introduction}

\begin{itemize}
    \item \blue{Background on the kagome lattice and physics.} 
    \item \blue{Nonlinear transport phenomena coming from the quantum metric} 
    \item \textcolor{purple}{Focus of the research: } \blue{Nonlinear transport phenomena 
    based on the quantum metric that could arise in bilayer magnetic kagome systems}

    \item \textcolor{purple}{Research gap/ question}: \blue{In nature, there are families 
    of kagome materials with: 1) spin-orbit coupling 2) magnetism, including various 
    magnetic orderings. The goal now is to shed light on quantum metric transport 
    in these systems, and see if there is something new one can learn from it. }
    
    
\end{itemize}

\section{Kagome Models}

\begin{itemize}
    \item \red{What is the magnetic space group for the system?} This has 
    been mentioned in \cite{herreraCornerModesBreathing2022}. For the breathing 
    kagome system, 
    The magnetic space group is \blue{p3m1}. The original system had $C_3$
    symmetry, which is broken by the kagome distortion. 
    To explain this, "3m" means that we have a three-fold rotation symmetry
    and mirror symmetry. 
    \item \red{Can we write a general TB model based on the knowledge of the MSG?}
    \item 
\end{itemize}


To motivate our discussion, we start with a monolayer system. The goal is to 
understand how to systematically break symmetries and write down models that 
obey the symmetry constraints. This has been discussed in
 \cite{geschnerBandTopologyBreathing2024}
First, we suppress all hoppings within the same sub-lattice. So, we have

\begin{equation}
   \mathcal{H} =  \begin{pmatrix}
    0 && \phi_1 && \phi_2 \\
    \phi_1^{\star} && 0 && \phi_3 \\
    \phi_2^{\star} && \phi_3^{\star} && 0
    \end{pmatrix}
\end{equation}


\section{Nonlinear transport from the quantum metric (BCP effect)}


\section{Methodology}

\green{Explanation of the numerical calculations being done and the simulation parameters}

\section{Results}

In this section, summarize the results pertaining to the trimerized model with broken 
$C_3$ symmetry 

\section{Discussion and Outlook}

%%%%%%%%%%%%%%%%%%%%%%%%%%%%%%%%%%%%%%%%%%%%%%%%%%%%%%%%%
%%%%%%%%%%%%%%%%%%%%%%%%%%%%%%%%%%%%%%%%%%%%%%%%%%%%%%%%%
%%%%%%%%%%%%%%%%%%%%%%%%%%%%%%%%%%%%%%%%%%%%%%%%%%%%%%%%%
\bibliographystyle{apsrev4-2}
\bibliography{bib_files/Kagome, bib_files/Transport}
\end{document}